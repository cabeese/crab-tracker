\documentclass[12pt]{article}

\setlength{\parindent}{0pt}
\setlength{\parskip}{2mm}

\usepackage{geometry}
 \geometry{
 letterpaper, left=20mm, right=20mm,  top=20mm,
 }
\usepackage{graphicx}
\graphicspath{ {graphics/} }
\usepackage{amssymb}
\usepackage[hidelinks]{hyperref}

%%%%%%%%%%%%%%%%%%%%%%%%%%%%%%%%%%%%%%%%%%%%%%%%%
\title{Crab Tracker - Technical Documentation}
%%%%%%%%%%%%%%%%%%%%%%%%%%%%%%%%%%%%%%%%%%%%%%%%%

\author{
	Noah Strong
}

\date{\today\ -- v1.0-wip}

\begin{document}

\maketitle

\tableofcontents{}

\section{Introduction}

The Crab Tracker project was designed as a cost-effective means of remotely
tracking crabs through acoustic signals.
Small piezoelectric transmitters can be waterproofed and attached to crabs,
and their intermittent signals can then be received by a set of four
piezoelectric receivers configured in a square array.

Because this product was built with very few ``off-the-shelf'' components,
it is somewhat complex and many of the finer details may be difficult for
future collaborators to infer based on the existing documentation.
This document aims to provide a technical overview of the project, including
the rationale for some of the design choices, in hopes of giving the reader
a deeper insight into the inner-workings of the product.

\section{Background and Overview}

While solutions already exist to aid in the tracking of aquatic animals,
they are often prohibitively expensive without significant financial resources.
The Crab Tracker product has been designed with cost and simplicity in mind,
so a majority of the components, including both hardware and software, are
custom-made.
However, much of the product, especially the software, has been designed
in a ``modular'' fashion, meaning that the various components are not tightly
coupled.
Simply put, one should be able to change out various components with relative
ease.

There are several major components of this project, each detailed in their
respective sections. Those components include:
\begin{itemize}
	\item The piezoelectric transmitters, which are to be attached to the
	      crabs themselves, and the piezoelectric receivers, which are to
	      be attached to the water vessel (such as a kayak) in a fixed shape
	      relative to each other. These are detailed in
	      Section \ref{sec:ee-hardware}.
	\item The receiving and calculating computers, which detect incoming
	      data from the receivers and process that data, eventually displaying
	      those results to the user.
	      These hardware components are detailed in
	      Section \ref{sec:cs-hardware} and the software is detailed in
	      Section \ref{sec:software}.
\end{itemize}

\section{Electronics Hardware}\label{sec:ee-hardware}

\section{Computer Hardware}\label{sec:cs-hardware}

\section{Software}\label{sec:software}

\end{document}

